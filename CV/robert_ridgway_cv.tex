%%%%%%%%%%%%%%%%%%%%%%%%%%%%%%%%%%%%%%%%%
% Medium Length Graduate Curriculum Vitae
% LaTeX Template
% Version 1.1 (9/12/12)
%
% This template has been downloaded from:
% http://www.LaTeXTemplates.com
%
% Original author:
% Rensselaer Polytechnic Institute (http://www.rpi.edu/dept/arc/training/latex/resumes/)
%
% Important note:
% This template requires the res.cls file to be in the same directory as the
% .tex file. The res.cls file provides the resume style used for structuring the
% document.
%
%%%%%%%%%%%%%%%%%%%%%%%%%%%%%%%%%%%%%%%%%

%----------------------------------------------------------------------------------------
%	PACKAGES AND OTHER DOCUMENT CONFIGURATIONS
%----------------------------------------------------------------------------------------

\documentclass[letter, margin, 10pt]{res} % Use the res.cls style, the font size can be changed to 11pt or 12pt here
\usepackage[linkcolor=black,colorlinks=true,urlcolor=cyan]{hyperref}
\usepackage{helvet} % Default font is the helvetica postscript font
%\usepackage{newcent} % To change the default font to the new century schoolbook postscript font uncomment this line and comment the one above
\usepackage{enumitem}

\topmargin=-0.7in
\oddsidemargin -.6in
\resumewidth = 6.9in
\addtolength{\textheight}{1.2in}

\setlength{\textwidth}{6.1in} % Text width of the document

\begin{document}

%----------------------------------------------------------------------------------------
%	NAME AND ADDRESS SECTION
%----------------------------------------------------------------------------------------

\moveleft.5\hoffset\centerline{\large\bf  Robert Ridgway} % Your name at the top
 
\moveleft\hoffset\vbox{\hrule width\resumewidth height 1pt}\smallskip % Horizontal line after name; adjust line thickness by changing the '1pt'
 
\moveleft.5\hoffset\centerline{University of Exeter} % Your address
\moveleft.5\hoffset\centerline{Physics Building, Stocker Road, Exeter, EX4 4QL}
\moveleft.5\hoffset\centerline{rr364@exeter.ac.uk \href{https://orcid.org/0000-0001-5534-0561}{ORCID: 0000-0001-5534-0561}}
\moveleft.5\hoffset\centerline{\href{https://RobertRidgway.github.io}{RobertRidgway.github.io}, \href{https://twitter.com/robbieridgway}{@robbieridgway} --- Updated August 2021}

%----------------------------------------------------------------------------------------
\begin{resume}

%----------------------------------------------------------------------------------------
%	OBJECTIVE SECTION
%----------------------------------------------------------------------------------------
 
% \section{RESEARCH INTERESTS}  

% My main interest in astrophysics involves the understanding of early star formation, from cloud core through to cluster evolution along the pre-main-sequence. Specifically, the application of robust statistical methodology to investigations to explore and explain these complex astrophysical processes. Understanding the consequences of the way we observe the universe and characterising how our assumptions influence our conclusions also drive my interests in the field.

%----------------------------------------------------------------------------------------
%	EDUCATION SECTION
%----------------------------------------------------------------------------------------
\vspace{-9pt}
\section{EDUCATION}

{\sl PhD in Physics}, University of Exeter, 2018-Present --- Supervisor: Prof. Nathan Mayne\\
\begin{itemize}[noitemsep,topsep=0pt,parsep=0pt,partopsep=0pt]
\item Used the Met Office Unified Model (UM) to look at the climate of exoplanets in 3D
\item Combined a chemical kinetics scheme with a photolysis scheme and the UM to look at the effects of stellar flares on terrestrial planets in 3D
\end{itemize}
\vspace{-10pt}
{\sl Master of Science in Space Physics}, University of Calgary, 2015-2018 --- Supervisor: Prof. Brian Jackel
\begin{itemize}[noitemsep,topsep=0pt,parsep=0pt,partopsep=0pt]
\item Analysis of the usage of travel-time magnetoseismology to construct density profiles of the near-Earth plasma environment
\item Used magnetometer data from the GOES and THEMIS spacecraft to look at determining the relative travel-times of signals  through the magnetosphere
\end{itemize}
\vspace{-10pt}
{\sl P.U.R.E. Studentship}, University of Calgary, 2014 --- Supervisor: Prof. Rene Plume\\
\null\quad\quad Awarded P.U.R.E. Studentship
\begin{itemize}[noitemsep,topsep=0pt,parsep=0pt,partopsep=0pt]
\item Work on characterising the D/H ratio of star-forming regions of the Orion Nebula using data from Herschel
\end{itemize}
\vspace{-10pt}
{\sl Bachelor of Science in Astrophysics} Honours First Class, University of Calgary, 2011-2015



%----------------------------------------------------------------------------------------
%	PUBLICATION SECTION
%----------------------------------------------------------------------------------------

\section{CO-AUTHOR PUBLICATIONS}

Benjamin Drummond, Eric Hebrard, Nathan J. Mayne, Olivia Venot, \textbf{Robert J. Ridgway}, Quentin Changeat, Shang-Min Tsai, James Manners, Pascal Tremblin, Nathan Luke Abraham, David Sing, and Krisztian Kohary. Implications of three-dimensional chemical transport in hot Jupiter atmospheres: Results from a consistently coupled chemistry-radiation-hydrodynamics model. Astronomy \& Astrophysics, 636:A68, April 2020. ISSN 0004-6361. doi:10.1051/0004-6361/201937153

Ian A. Boutle, Manoj Joshi, F. Hugo Lambert, Nathan J. Mayne, Duncan Lyster, James Manners, \textbf{Robert Ridgway}, and Krisztian Kohary. Mineral dust increases the habitability of terrestrial planets but confounds biomarker detection. Nature Communications 11, 2731, June 2020. ISSN 2041-1723. doi:10.1038/s41467-020-16543-8

Jake K. Eager, David J. Reichelt, Nathan J. Mayne, F. Hugo Lambert, Denis E. Sergeev, \textbf{Robert J. Ridgway}, James Manners, Ian A. Boutle, Timothy M. Lenton, and Krisztian Kohary. Implications of different stellar spectra for the climate of tidally locked Earth-like exoplanets. Astronomy \& Astrophysics, 639:A99, July 2020. ISSN 0004-6361. doi:10.1051/0004-6361/202038089



%----------------------------------------------------------------------------------------
%	TALKS SECTION
%----------------------------------------------------------------------------------------
\parskip \baselineskip
\vspace{-6pt}
\section{SCIENTIFIC TALKS \& CONFERENCES}
1 contributed conference talk, 2 contributed conference posters.

\vspace{-4pt}

April 2021, UK Exoplanet Community Meeting (UKEXOM) 2021, Contributed Talk\\
December 2016, American Geophysical Union (AGU) Fall Meeting, Contributed Poster\\
June 2015, Canadian Association of Physicists (CAP) Congress, Contributed Poster

\section{Competitive Scholarships and Awards}
\vspace{4pt}
Alberta Graduate Student Scholarship - \$3000 CAD\hfill\llap{2017}\\
Queen Elizabeth II Graduate Scholarship (Master's) - \$3600 CAD\hfill\llap{2016}\\
Queen Elizabeth II Graduate Scholarship (Master's) - \$10800 CAD\hfill\llap{2016}\\
University of Calgary Undergraduate Merit Award - \$750 CAD\hfill\llap{2014}\\
P.U.R.E. (Program for Undergraduate Research Experience) - \$6000 CAD\hfill\llap{2014}\\
Jason Lang Scholarship - \$1000 CAD (x3)\hfill\llap{2012, 2013, 2014}\\
Alexander Rutherford Scholarship - \$2500 CAD\hfill\llap{2011}\\
University of Calgary Entrance Scholarship - \$1250 CAD\hfill\llap{2011}
\section{TEACHING EXPERIENCE}

{\sl Undergraduate Teaching Assistant}, 2015-2017
\begin{itemize}[noitemsep,topsep=0pt,parsep=0pt,partopsep=0pt]
\item Assisted in teaching of 20-30 second year undergraduates in physics labs and computer science 
\item Demonstrated use of UNIX commands, analysis of experimental results, scientific use of Python, \& report writing
%\item Duties included report assessment and feedback
\end{itemize}



% Awards - summer placement, 2012 ~£1500
 
%----------------------------------------------------------------------------------------
%	REFERENCES SECTION
%----------------------------------------------------------------------------------------

% \section{REFERENCES} 

% Dr Louis-Gregory Strolger, Space Telescope Science Institute, strolger@stsci.edu

% Prof. Tim Naylor, University of Exeter, timn@astro.ex.ac.uk

% Dr Jennifer Hatchell, University of Exeter, hatchell@astro.ex.ac.uk

 

\end{resume}
\end{document}




